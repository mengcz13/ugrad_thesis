\chapter{结论与展望}
\label{cha:conclusion}
开放通道SSD因其能够将内部的结构暴露给外部程序从而提供更大的灵活性和更多的优化空间,得到了越来越广泛的应用,尤其是在数据库领域已经有了一些工作成果。但基于开放通道SSD优化高性能应用的工作目前还不多见。本文据此从两种高性能应用的负载特征分析入手,针对其特征实现了采用超级块映射和贪心垃圾回收方法的IO优化策略,相对于已有方法实现了接近的IO吞吐性能和更低的映射表维护开销。

本文利用高性能应用的IO负载中覆盖写比例较高的特点,判断其IO过程中会在SSD上产生大量不含任何脏页的无效数据块,从而将现有方法的映射粒度从页级别提升到超级块级别,在维持原有IO吞吐性能的同时降低了映射表的维护成本。本文同时尝试了模仿日志结构文件系统改进垃圾回收方式的做法,结果表明这一做法在垃圾回收时重新写入脏页的成本过高,写放大系数偏大,因而并不适用于本文所研究的高性能应用负载。

由于时间和精力所限,本文涉及的IO优化策略仅覆盖了地址映射、垃圾回收和负载均衡三方面的内容,实际系统中还可以通过写缓存、使用后台线程进行垃圾回收等方式进一步提高系统性能。这部分可以作为后续的研究内容。